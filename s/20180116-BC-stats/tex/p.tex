\documentclass[onepage]{beamer}
\mode<presentation>{
	\setbeamercovered{transparent}
	%\beamertemplatenavigationsymbolsempty
	\setbeamertemplate{footline}[frame number]
% 	\usefonttheme{professionalfonts}
}

% https://tex.stackexchange.com/questions/34166/understanding-minipages-aligning-at-top
\usepackage{adjustbox}

% https://tex.stackexchange.com/questions/124256/how-do-i-get-numbered-entries-in-a-beamer-bibliography
\setbeamertemplate{bibliography item}{\insertbiblabel}

% https://tex.stackexchange.com/questions/49048/how-to-cite-one-bibentry-in-full-length-in-the-body-text
\usepackage{bibentry}
\bibliographystyle{plain}
\nobibliography*
%
% https://tex.stackexchange.com/questions/163827/wrong-vertical-spaces-using-bibentry-within-beamer/163842
\def\mybeamernewblock{%
  \usebeamercolor[fg]{bibliography entry author}%
  \usebeamerfont{bibliography entry author}%
  \usebeamertemplate{bibliography entry author}%
  \def\newblock{%
    \usebeamercolor[fg]{bibliography entry title}%
    \usebeamerfont{bibliography entry title}%
    \usebeamertemplate{bibliography entry title}%
    \def\newblock{%
      \usebeamercolor[fg]{bibliography entry location}%
      \usebeamerfont{bibliography entry location}%
      \usebeamertemplate{bibliography entry location}%
      \def\newblock{%
        \usebeamercolor[fg]{bibliography entry note}%
        \usebeamerfont{bibliography entry note}%
        \usebeamertemplate{bibliography entry note}}}}%
  \leavevmode
}
\newenvironment{references}{\begin{itemize}\let\newblock\mybeamernewblock}{\end{itemize}}


\setbeamersize{text margin left=10pt, text margin right=10pt}
\setbeamertemplate{itemize items}[circle]

\beamertemplatenavigationsymbolsempty

% http://tex.stackexchange.com/questions/8680/how-can-i-insert-a-newline-in-a-framebox
%\usepackage{minibox}
%\usepackage{framed}
%\usepackage[usestackEOL]{stackengine}

% % http://tex.stackexchange.com/questions/167000/annotating-tables-with-tikz-adding-arrows
% \usepackage{color, colortbl}
\usepackage{tikz}
\tikzstyle{every picture}+=[remember picture]
%\usetikzlibrary{tikzmark, positioning, fit, shapes.misc}

%% http://tex.stackexchange.com/questions/91124/itemize-removing-natural-indent
%\usepackage{enumitem}

%% http://tex.stackexchange.com/questions/41408/a-five-level-deep-list
%\usepackage{enumitem}
%\setlistdepth{9}

% https://tex.stackexchange.com/questions/20792/how-to-superimpose-latex-on-a-picture
\usepackage{overpic}

% % http://tex.stackexchange.com/questions/32661/how-to-locate-figures-with-x-y-specified-location-in-a-presentation
% \usepackage[absolute,overlay]{textpos} % absolute positioning of stuff
% \setlength{\TPHorizModule}{1mm}
% \setlength{\TPVertModule}{1mm}

\usepackage{graphicx}
\graphicspath{{../img/}}
\AtBeginDocument{\DeclareGraphicsExtensions{.eps, .png, .gif, .pdf, .jpg}}

\usepackage[makeroom]{cancel}

\usepackage[english]{babel}
\usepackage[T1]{fontenc}
\usepackage{times}

% \usepackage{amssymb}
% \usepackage{nicefrac}
% \usepackage{bbm}
% \usepackage{esint}
% \usepackage{sidecap}

\usepackage{hyperref}
% \hypersetup{pdfpagemode=FullScreen}

\newcommand{\HIDE}[1]{}

\newcommand{\skipline}{{\ }\\}

\newcommand{\EMAIL}{{\color{blue}randreev{\tiny\color{white}.\hspace{-1.5pt}}@{\tiny\color{white}.\hspace{-1.5pt}}stat.sinica.edu.tw}}

\author{\small RA}
\subject{Talks}

\newcommand{\CITE}[1]{{\footnotesize[#1]}}

% \input{definitions}

\providecommand{\DIV}{\mathop{\text{div}}}
\providecommand{\GRAD}{\mathop{\text{grad}}}

\providecommand{\IE}{\mathbb{E}}
\providecommand{\IP}{\mathbb{P}}
\providecommand{\IR}{\mathbb{R}}
\providecommand{\IZ}{\mathbb{Z}}

\providecommand{\duality}[2]{\langle #1 \rangle_{#2}}
\providecommand{\norm}[2]{\| #1 \|_{#2}}
\providecommand{\seminorm}[2]{| #1 |_{#2}}
\providecommand{\VERT}{\ensuremath{| \! | \! |}}
\newcommand{\tnorm}[2]{\VERT{#1}\VERT_{{#2}}}

\newcommand{\cA}{\mathcal{A}}
\newcommand{\cB}{\mathcal{B}}
\newcommand{\cL}{\mathcal{L}}
\newcommand{\cN}{\mathcal{N}}
\newcommand{\cT}{\mathcal{T}}
\newcommand{\cX}{\mathcal{X}}
\newcommand{\cY}{\mathcal{Y}}

\providecommand{\Abf}{\mathbf{A}}
\providecommand{\Bbf}{\mathbf{B}}
\providecommand{\Dbf}{\mathbf{D}}
\providecommand{\Ibf}{\mathbf{I}}
\providecommand{\Jbf}{\mathbf{J}}
\providecommand{\Fbf}{\mathbf{F}}
\providecommand{\Hbf}{\mathbf{H}}
\providecommand{\Mbf}{\mathbf{M}}
\providecommand{\Tbf}{\mathbf{T}}
\providecommand{\Pbf}{\mathbf{P}}
\providecommand{\Vbf}{\mathbf{V}}
\providecommand{\pbf}{\mathbf{p}}
\providecommand{\ubf}{\mathbf{u}}
\providecommand{\vbf}{\mathbf{v}}
\providecommand{\wbf}{\mathbf{w}}
\providecommand{\ybf}{\mathbf{y}}
\providecommand{\zbf}{\mathbf{z}}

\renewcommand{\vec}[1]{\mathbf{#1}}

\providecommand{\T}{\mathsf{T}}

\renewcommand{\hat}[1]{\widehat{#1}}
\renewcommand{\tilde}[1]{\widetilde{#1}}

\newcommand{\rd}{\,\mathrm{d}}

\newcommand{\TEXT}[1]{\quad\text{#1}\quad}

% http://tex.stackexchange.com/questions/211518/beamer-vfill-and-itemize
\def\Bottom#1{\vskip 0pt plus 1filll #1}
\def\BottomRight#1{\Bottom{\hfill #1}}

% MATLAB CODE LISTING
\usepackage{color}
\definecolor{DarkBlue}{rgb}{0,0,0.4}
% \definecolor{DarkRed}{rgb}{0.3,0,0}
\definecolor{DarkGreen}{rgb}{0,0.3,0}
\usepackage{listings}
\lstset{%
	language=Python,
	basicstyle=\bf\ttfamily\footnotesize,
	keywordstyle=\color{DarkBlue},
	numbers=left, numberstyle=\footnotesize, numbersep=4pt,
	commentstyle={\color{DarkGreen}},
% 	backgroundcolor=\color{white},
	showspaces=false, showstringspaces=false, showtabs=false,
	frame=none,
	tabsize=4,
	breaklines=true, breakatwhitespace=false,
	emphstyle={[1]\color{blue}},
	emphstyle={[2]\color{DarkGreen}},
	%morekeywords={parfor,true,false},
	xleftmargin=8pt,
	numbers=none
}


%%%%%%%%%%%%%%%%%%%%%%%%%%%%%%%%%%%%%%%%%%%%%%%%%%%%%%%%%%%%%%%%%%%%%%%%%%%%%%%%
%%
%%%%%%%%%%%%%%%%%%%%%%%%%%%%%%%%%%%%%%%%%%%%%%%%%%%%%%%%%%%%%%%%%%%%%%%%%%%%%%%%

\usepackage{ifthen}

\newcommand{\REDBOX}[1]{
	\setlength{\fboxrule}{1pt}
	\fcolorbox{red}{SeeMeBarely}{$\displaystyle
		#1
	$}
}

\definecolor{SeeMeBarely}{RGB}{230,230,230}
\definecolor{Purple}{RGB}{128,0,128}
\definecolor{DeepPurple}{RGB}{32,0,96}
\newcommand{\ra}[1]{{\color{blue}{#1}}}
\newcommand{\cred}[1]{{\color{red}{#1}}}
\newcommand{\cblu}[1]{{\color{blue}{#1}}}
\newcommand{\cpur}[1]{{\color{Purple}{#1}}}

\DeclareMathOperator*{\argmin}{arg\,min}

\newcommand{\ItemComment}[1]{\hfill{\scriptsize(#1)\normalsize}}


% http://www.webnots.com/vibgyor-rainbow-color-codes/
\definecolor{a}{RGB}{148, 0, 211}
\definecolor{b}{RGB}{75, 0, 130}
\definecolor{c}{RGB}{0, 0, 255}
\definecolor{d}{RGB}{0, 160, 0}
\definecolor{e}{RGB}{200, 200, 0}
\definecolor{f}{RGB}{255, 127, 0}
\definecolor{g}{RGB}{255, 0, 0}
%
\definecolor{z}{RGB}{0, 0, 0}
\definecolor{w}{RGB}{255, 255, 255}

% http://tex.stackexchange.com/questions/17611/how-does-one-type-chinese-in-latex
\usepackage{CJKutf8}
\AtBeginDvi{\input{zhwinfonts}}
%
\newcommand{\REN}{\begin{CJK*}{UTF8}{gbsn}人\end{CJK*}}
\newcommand{\ren}[1]{{\color{#1}\REN}}



%%%%%%%%%%%%%%%%%%%%%%%%%%%%%%%%%%%%%%%%%%%%%%%%%%%%%%%%%%%%%%%%%%%%%%%%%%%%%%%%
\begin{document}
%%%%%%%%%%%%%%%%%%%%%%%%%%%%%%%%%%%%%%%%%%%%%%%%%%%%%%%%%%%%%%%%%%%%%%%%%%%%%%%%
%%%%%%%%%%%%%%%%%%%%%%%%%%%%%%%%%%%%%%%%%%%%%%%%%%%%%%%%%%%%%%%%%%%%%%%%%%%%%%%%

%%%%%%%%%%%%%%%%%%%%%%%%%%%%%%%%%%%%%%%%%%%%%%%%%%%%%%%%%%%%%%%%%%%%%%%%%%%%%%%%
\section{Intro}
%%%%%%%%%%%%%%%%%%%%%%%%%%%%%%%%%%%%%%%%%%%%%%%%%%%%%%%%%%%%%%%%%%%%%%%%%%%%%%%%



\begin{frame}[plain,t]
	\begin{center}
		%\small
		%
		Some stats on the GSE75688 BC dataset
		%
		\\[1\baselineskip]
		\small
		RA
% 		\\[1\baselineskip]
% 		\footnotesize
% 		ISS, AS \\ \EMAIL

		\vspace{1cm}

		%

	\end{center}

	\Bottom{
		\scriptsize
		%Support: 
		\hfill
		Jan 16, 2018
		\\ {\ }
	}
\end{frame}


%%%

%%%

%%%

\begin{frame}[t]{Clustering index -- overview}{}
	\begin{center}
		\includegraphics<1>[width=0.99\textwidth]{B_go-by-wq/ci-vs-sz_1}
		\includegraphics<2>[width=0.99\textwidth]{B_go-by-wq/ci-vs-sz_2}
	\end{center}
\end{frame}

%%%

\begin{frame}[t]{GO terms ordered by the windowed quantile}{}
	\begin{table}[]
		\tiny
		\resizebox{\columnwidth}{!}{%
		\begin{tabular}{rlllll}
		      & GO ID      & GO name                                               & Size & Clustering index & CI quantile  \\
		1     & GO:1901339 & regulation of store-operated calcium channel activity & 1    & -0.9958762887    & 0.0001906214 \\
		2     & GO:0005132 & type I interferon receptor binding                    & 6    & -0.9821958457    & 0.0004095004 \\
		3     & GO:0098633 & collagen fibril binding                               & 1    & -0.9954545455    & 0.0004765536 \\
		4     & GO:1904027 & negative regulation of collagen fibril organization   & 1    & -0.9954545455    & 0.0004765536 \\
		5     & GO:0015874 & norepinephrine transport                              & 3    & -0.9875518672    & 0.0005292872 \\
		6     & GO:0005333 & norepinephrine transmembrane transporter activity     & 3    & -0.9875518672    & 0.0005292872 \\
		7     & GO:0008188 & neuropeptide receptor activity                        & 12   & -0.8926174497    & 0.0005382131 \\
		8     & GO:0004956 & prostaglandin D receptor activity                     & 2    & -0.9947089947    & 0.0005512679 \\
		9     & GO:0001785 & prostaglandin J receptor activity                     & 2    & -0.9947089947    & 0.0005512679 \\
		10    & GO:0014050 & negative regulation of glutamate secretion            & 5    & -0.9791122715    & 0.0006347191 \\
		11    & GO:0003837 & beta-ureidopropionase activity                        & 1    & -0.9953488372    & 0.0007624857 \\
		12    & GO:0033038 & bitter taste receptor activity                        & 19   & -0.8965517241    & 0.0007680492 \\
		13    & GO:0005590 & collagen type VII trimer                              & 1    & -0.9951219512    & 0.0009531071 \\
		...   &            &                                                       &      &                  &              \\
		17483 & GO:0070180 & large ribosomal subunit rRNA binding                  & 1    & 0.0544090056     & 1            \\
		17484 & GO:0099503 & secretory vesicle                                     & 3    & -0.4171220401    & 1            \\
		17485 & GO:0030235 & nitric-oxide synthase regulator activity              & 6    & -0.1730418944    & 1            \\
		17486 & GO:0071364 & cellular response to epidermal growth factor stimulus & 40   & 0.1985428051     & 1            \\
		17487 & GO:0016310 & phosphorylation                                       & 679  & 0.7559198543     & 1            \\
		17488 & GO:0016740 & transferase activity                                  & 1717 & 0.8214936248     & 1            \\
		17489 & GO:0016020 & membrane                                              & 7464 & 0.8178506375     & 1           
		\end{tabular}
		}
	\end{table}
\end{frame}

%%%

\begin{frame}[t]{Transformation to ``GO space''}{}
	In ``GO space'', 
	\begin{itemize}
	\item
		the coordinates are the GO terms
	\item
		whose value is the sum of gene expression in that GO category.
	\end{itemize}
	
\end{frame}

%%%

\begin{frame}[t]{Transformation to ``GO space''}{}
	\begin{center}
		\includegraphics<01>[width=0.9\textwidth]{C_goordinates/CI-vs-NxGO}
	\end{center}
\end{frame}

%%%

\begin{frame}[t]{Transformation to ``GO space''}{}
	First 
	{\only<01>{60}}%
	{\only<02->{20}}
	GO-ordinates
	\begin{center}
		\includegraphics<01>[width=0.9\textwidth]{C_goordinates/tsne_dim=60_run=1_sub=all}
		\includegraphics<02>[width=0.9\textwidth]{C_goordinates/tsne_dim=20_run=1_sub=all}
		
		\includegraphics<03>[width=0.9\textwidth]{C_goordinates/tsne_dim=20_run=1_sub=0}
		\includegraphics<04>[width=0.9\textwidth]{C_goordinates/tsne_dim=20_run=1_sub=1}
		\includegraphics<05>[width=0.9\textwidth]{C_goordinates/tsne_dim=20_run=1_sub=2}
		\includegraphics<06>[width=0.9\textwidth]{C_goordinates/tsne_dim=20_run=1_sub=3}
		\includegraphics<07>[width=0.9\textwidth]{C_goordinates/tsne_dim=20_run=1_sub=4}
		\includegraphics<08>[width=0.9\textwidth]{C_goordinates/tsne_dim=20_run=1_sub=5}
		\includegraphics<09>[width=0.9\textwidth]{C_goordinates/tsne_dim=20_run=1_sub=6}
		\includegraphics<10>[width=0.9\textwidth]{C_goordinates/tsne_dim=20_run=1_sub=7}
		\includegraphics<11>[width=0.9\textwidth]{C_goordinates/tsne_dim=20_run=1_sub=8}
		\includegraphics<12>[width=0.9\textwidth]{C_goordinates/tsne_dim=20_run=1_sub=9}
		\includegraphics<13>[width=0.9\textwidth]{C_goordinates/tsne_dim=20_run=1_sub=10}
		\includegraphics<14>[width=0.9\textwidth]{C_goordinates/tsne_dim=20_run=1_sub=11}
		\includegraphics<15>[width=0.9\textwidth]{C_goordinates/tsne_dim=20_run=1_sub=12}
		\includegraphics<16>[width=0.9\textwidth]{C_goordinates/tsne_dim=20_run=1_sub=13}
	\end{center}
\end{frame}

%%%

\begin{frame}[t]{Tumor classifier based on ``DNA-replication''}{}
	\begin{center}
		\includegraphics<01>[width=0.9\textwidth]{9_tree/tree_Top50_c}
	\end{center}
\end{frame}

%%%

\begin{frame}[t,fragile]{Tumor classifier based on ``DNA-replication''}{}
\begin{lstlisting}
m = Sequential([
	Dropout(rate=0.6, input_shape=(num_genes,)),
	Dense(num_classes, kernel_regularizer=L1L2(1, 1)),
	Activation('softmax'),
])

m.compile(loss='categorical_crossentropy', optimizer='adam')

m.fit(X, Y, epochs=9000)
\end{lstlisting}
\end{frame}

%%%

\begin{frame}[t]{Tumor classifier based on ``DNA-replication''}{}
	\begin{center}
		\includegraphics<01>[width=0.9\textwidth]{D_classifier-nn/samples}
		\includegraphics<02>[width=0.9\textwidth]{D_classifier-nn/confusion}
	\end{center}
\end{frame}

%%

\begin{frame}[t]{Transformation to ``GO space''}{}
	First 20 GO-ordinates
	\begin{center}
		\includegraphics<01>[width=0.9\textwidth]{C_goordinates-D/tsne_dim=20_run=1_sub=1}
		\includegraphics<02>[width=0.9\textwidth]{C_goordinates-D/tsne_dim=20_run=1_sub=2}
		\includegraphics<03>[width=0.9\textwidth]{C_goordinates-D/tsne_dim=20_run=1_sub=3}
		\includegraphics<04>[width=0.9\textwidth]{C_goordinates-D/tsne_dim=20_run=1_sub=4}
		\includegraphics<05>[width=0.9\textwidth]{C_goordinates-D/tsne_dim=20_run=1_sub=5}
		\includegraphics<06>[width=0.9\textwidth]{C_goordinates-D/tsne_dim=20_run=1_sub=6}
		\includegraphics<07>[width=0.9\textwidth]{C_goordinates-D/tsne_dim=20_run=1_sub=7}
		\includegraphics<08>[width=0.9\textwidth]{C_goordinates-D/tsne_dim=20_run=1_sub=8}
		\includegraphics<09>[width=0.9\textwidth]{C_goordinates-D/tsne_dim=20_run=1_sub=9}
		\includegraphics<10>[width=0.9\textwidth]{C_goordinates-D/tsne_dim=20_run=1_sub=10}
		\includegraphics<11>[width=0.9\textwidth]{C_goordinates-D/tsne_dim=20_run=1_sub=11}
		\includegraphics<12>[width=0.9\textwidth]{C_goordinates-D/tsne_dim=20_run=1_sub=12}
		\includegraphics<13>[width=0.9\textwidth]{C_goordinates-D/tsne_dim=20_run=1_sub=13}
		\includegraphics<14>[width=0.9\textwidth]{C_goordinates-D/tsne_dim=20_run=1_sub=all}
	\end{center}
\end{frame}


%%%%%%%%%%%%%%%%%%%%%%%%%%%%%%%%%%%%%%%%%%%%%%%%%%%%%%%%%%%%%%%%%%%%%%%%%%%%%%%%%
%\section{Extra}
%%%%%%%%%%%%%%%%%%%%%%%%%%%%%%%%%%%%%%%%%%%%%%%%%%%%%%%%%%%%%%%%%%%%%%%%%%%%%%%%%
%
%
\newcounter{finalframe}
\setcounter{finalframe}{\value{framenumber}}
% Backup frames follow
%
%
%\begin{frame}
%	Appendix
%\end{frame}
%
%%
%
%\begin{frame}
%	%
%\end{frame}
%
%
% FINAL SLIDE
\setbeamercolor{background canvas}{bg=black}
\begin{frame}[plain,b]
	\hfill
	\tiny
	\color{gray}
	this slide is intentionally left blank
\end{frame}
\setbeamercolor{background canvas}{bg=white}


%%%%%%%%%%%%%%%%%%%%%%%%%%%%%%%%%%%%%%%%%%%%%%%%%%%%%%%%%%%%%%%%%%%%%%%%%%%%%%%%%
%\section{Bibliography}
%%%%%%%%%%%%%%%%%%%%%%%%%%%%%%%%%%%%%%%%%%%%%%%%%%%%%%%%%%%%%%%%%%%%%%%%%%%%%%%%%

% {
% \tiny
% \bibliography{../../../r/refs}
% }


%%%%%%%%%%%%%%%%%%%%%%%%%%%%%%%%%%%%%%%%%%%%%%%%%%%%%%%%%%%%%%%%%%%%%%%%%%%%%%%%
\setcounter{framenumber}{\value{finalframe}}
\end{document}
%%%%%%%%%%%%%%%%%%%%%%%%%%%%%%%%%%%%%%%%%%%%%%%%%%%%%%%%%%%%%%%%%%%%%%%%%%%%%%%%
%%%%%%%%%%%%%%%%%%%%%%%%%%%%%%%%%%%%%%%%%%%%%%%%%%%%%%%%%%%%%%%%%%%%%%%%%%%%%%%%

