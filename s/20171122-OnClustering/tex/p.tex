\documentclass[onepage]{beamer}
\mode<presentation>{
	\setbeamercovered{transparent}
	%\beamertemplatenavigationsymbolsempty
	\setbeamertemplate{footline}[frame number]
% 	\usefonttheme{professionalfonts}
}

% https://tex.stackexchange.com/questions/34166/understanding-minipages-aligning-at-top
\usepackage{adjustbox}

% https://tex.stackexchange.com/questions/49048/how-to-cite-one-bibentry-in-full-length-in-the-body-text
\usepackage{bibentry}
\bibliographystyle{plain}
\nobibliography*
%
% https://tex.stackexchange.com/questions/163827/wrong-vertical-spaces-using-bibentry-within-beamer/163842
\def\mybeamernewblock{%
  \usebeamercolor[fg]{bibliography entry author}%
  \usebeamerfont{bibliography entry author}%
  \usebeamertemplate{bibliography entry author}%
  \def\newblock{%
    \usebeamercolor[fg]{bibliography entry title}%
    \usebeamerfont{bibliography entry title}%
    \usebeamertemplate{bibliography entry title}%
    \def\newblock{%
      \usebeamercolor[fg]{bibliography entry location}%
      \usebeamerfont{bibliography entry location}%
      \usebeamertemplate{bibliography entry location}%
      \def\newblock{%
        \usebeamercolor[fg]{bibliography entry note}%
        \usebeamerfont{bibliography entry note}%
        \usebeamertemplate{bibliography entry note}}}}%
  \leavevmode
}
\newenvironment{references}{\begin{itemize}\let\newblock\mybeamernewblock}{\end{itemize}}


\setbeamersize{text margin left=10pt, text margin right=10pt}
\setbeamertemplate{itemize items}[circle]

\beamertemplatenavigationsymbolsempty

% http://tex.stackexchange.com/questions/8680/how-can-i-insert-a-newline-in-a-framebox
%\usepackage{minibox}
%\usepackage{framed}
%\usepackage[usestackEOL]{stackengine}

% % http://tex.stackexchange.com/questions/167000/annotating-tables-with-tikz-adding-arrows
% \usepackage{color, colortbl}
\usepackage{tikz}
\tikzstyle{every picture}+=[remember picture]
%\usetikzlibrary{tikzmark, positioning, fit, shapes.misc}

%% http://tex.stackexchange.com/questions/91124/itemize-removing-natural-indent
%\usepackage{enumitem}

%% http://tex.stackexchange.com/questions/41408/a-five-level-deep-list
%\usepackage{enumitem}
%\setlistdepth{9}

% https://tex.stackexchange.com/questions/20792/how-to-superimpose-latex-on-a-picture
\usepackage{overpic}

% % http://tex.stackexchange.com/questions/32661/how-to-locate-figures-with-x-y-specified-location-in-a-presentation
% \usepackage[absolute,overlay]{textpos} % absolute positioning of stuff
% \setlength{\TPHorizModule}{1mm}
% \setlength{\TPVertModule}{1mm}

\usepackage{graphicx}
\graphicspath{{../img/}}
\AtBeginDocument{\DeclareGraphicsExtensions{.eps, .png, .gif, .pdf}}

\usepackage[makeroom]{cancel}

\usepackage[english]{babel}
\usepackage[T1]{fontenc}
\usepackage{times}

% \usepackage{amssymb}
% \usepackage{nicefrac}
% \usepackage{bbm}
% \usepackage{esint}
% \usepackage{sidecap}

\usepackage{hyperref}
% \hypersetup{pdfpagemode=FullScreen}

\newcommand{\HIDE}[1]{}

\newcommand{\skipline}{{\ }\\}

\newcommand{\EMAIL}{{\color{blue}randreev{\tiny\color{white}.\hspace{-1.5pt}}@{\tiny\color{white}.\hspace{-1.5pt}}stat.sinica.edu.tw}}

\author{\small RA}
\subject{Talks}

\newcommand{\CITE}[1]{{\footnotesize[#1]}}

% \input{definitions}

\providecommand{\DIV}{\mathop{\text{div}}}
\providecommand{\GRAD}{\mathop{\text{grad}}}

\providecommand{\IE}{\mathbb{E}}
\providecommand{\IP}{\mathbb{P}}
\providecommand{\IR}{\mathbb{R}}
\providecommand{\IZ}{\mathbb{Z}}

\providecommand{\duality}[2]{\langle #1 \rangle_{#2}}
\providecommand{\norm}[2]{\| #1 \|_{#2}}
\providecommand{\seminorm}[2]{| #1 |_{#2}}
\providecommand{\VERT}{\ensuremath{| \! | \! |}}
\newcommand{\tnorm}[2]{\VERT{#1}\VERT_{{#2}}}

\newcommand{\cA}{\mathcal{A}}
\newcommand{\cB}{\mathcal{B}}
\newcommand{\cL}{\mathcal{L}}
\newcommand{\cN}{\mathcal{N}}
\newcommand{\cT}{\mathcal{T}}
\newcommand{\cX}{\mathcal{X}}
\newcommand{\cY}{\mathcal{Y}}

\providecommand{\Abf}{\mathbf{A}}
\providecommand{\Bbf}{\mathbf{B}}
\providecommand{\Dbf}{\mathbf{D}}
\providecommand{\Ibf}{\mathbf{I}}
\providecommand{\Jbf}{\mathbf{J}}
\providecommand{\Fbf}{\mathbf{F}}
\providecommand{\Hbf}{\mathbf{H}}
\providecommand{\Mbf}{\mathbf{M}}
\providecommand{\Tbf}{\mathbf{T}}
\providecommand{\Pbf}{\mathbf{P}}
\providecommand{\Vbf}{\mathbf{V}}
\providecommand{\pbf}{\mathbf{p}}
\providecommand{\ubf}{\mathbf{u}}
\providecommand{\vbf}{\mathbf{v}}
\providecommand{\wbf}{\mathbf{w}}
\providecommand{\ybf}{\mathbf{y}}
\providecommand{\zbf}{\mathbf{z}}

\renewcommand{\vec}[1]{\mathbf{#1}}

\providecommand{\T}{\mathsf{T}}

\renewcommand{\hat}[1]{\widehat{#1}}
\renewcommand{\tilde}[1]{\widetilde{#1}}

\newcommand{\rd}{\,\mathrm{d}}

\newcommand{\TEXT}[1]{\quad\text{#1}\quad}

% http://tex.stackexchange.com/questions/211518/beamer-vfill-and-itemize
\def\Bottom#1{\vskip 0pt plus 1filll #1}
\def\BottomRight#1{\Bottom{\hfill #1}}

% MATLAB CODE LISTING
\usepackage{color}
\definecolor{DarkBlue}{rgb}{0,0,0.4}
% \definecolor{DarkRed}{rgb}{0.3,0,0}
\definecolor{DarkGreen}{rgb}{0,0.3,0}
\usepackage{listings}
\lstset{%
	language=MATLAB,
	basicstyle=\bf\ttfamily\tiny,%\footnotesize,
	keywordstyle=\color{DarkBlue},
	numbers=left, numberstyle=\footnotesize, numbersep=4pt,
	commentstyle={\color{DarkGreen}},
% 	backgroundcolor=\color{white},
	showspaces=false, showstringspaces=false, showtabs=false,
	frame=none,
	tabsize=4,
	breaklines=true, breakatwhitespace=false,
	emph={[1]femT_,femT_assemE,femT_assemF,femT_assemFE,femTX_assemLoad,spacetime},
	emphstyle={[1]\color{blue}},
	emph={[2]femX_,femX_MA,femX_b,femX_init,femX_show},
	emphstyle={[2]\color{DarkGreen}},
	morekeywords={parfor,true,false},
	xleftmargin=8pt,
	numbers=none
}


%%%%%%%%%%%%%%%%%%%%%%%%%%%%%%%%%%%%%%%%%%%%%%%%%%%%%%%%%%%%%%%%%%%%%%%%%%%%%%%%
%%
%%%%%%%%%%%%%%%%%%%%%%%%%%%%%%%%%%%%%%%%%%%%%%%%%%%%%%%%%%%%%%%%%%%%%%%%%%%%%%%%

\usepackage{ifthen}

\newcommand{\REDBOX}[1]{
	\setlength{\fboxrule}{1pt}
	\fcolorbox{red}{SeeMeBarely}{$\displaystyle
		#1
	$}
}

\definecolor{SeeMeBarely}{RGB}{230,230,230}
\definecolor{Purple}{RGB}{128,0,128}
\definecolor{DeepPurple}{RGB}{32,0,96}
\newcommand{\ra}[1]{{\color{blue}{#1}}}
\newcommand{\cred}[1]{{\color{red}{#1}}}
\newcommand{\cblu}[1]{{\color{blue}{#1}}}
\newcommand{\cpur}[1]{{\color{Purple}{#1}}}

\DeclareMathOperator*{\argmin}{arg\,min}

\newcommand{\ItemComment}[1]{\hfill{\scriptsize(#1)\normalsize}}

%\setbeamercolor{math text}{fg=DeepPurple}
%\setbeamercolor{inlined}{fg=DeepPurple}

% \setlength\abovedisplayskip{0pt}
% \setlength\belowdisplayskip{0pt}

% % http://tex.stackexchange.com/questions/269844/tikz-full-page-with-absolute-node-positioning
% \newcommand{\halfblack}[1]{%
% 	\begin{tikzpicture}[remember picture, overlay]%
% 		\node (back names) [shape=rectangle, fill=black, minimum height=#1\paperheight, minimum width=1.1\paperwidth, anchor=south] at (current page.south) {};%
% 	\end{tikzpicture}%
% }
% \newcommand{\halfwhite}[1]{%
% 	\begin{tikzpicture}[remember picture, overlay]%
% 		\node (back names) [shape=rectangle, fill=white, minimum height=#1\paperheight, minimum width=1.1\paperwidth, anchor=north] at (current page.north) {};%
% 	\end{tikzpicture}%
% }





% http://www.webnots.com/vibgyor-rainbow-color-codes/
\definecolor{a}{RGB}{148, 0, 211}
\definecolor{b}{RGB}{75, 0, 130}
\definecolor{c}{RGB}{0, 0, 255}
\definecolor{d}{RGB}{0, 160, 0}
\definecolor{e}{RGB}{200, 200, 0}
\definecolor{f}{RGB}{255, 127, 0}
\definecolor{g}{RGB}{255, 0, 0}
%
\definecolor{z}{RGB}{0, 0, 0}
\definecolor{w}{RGB}{255, 255, 255}

% http://tex.stackexchange.com/questions/17611/how-does-one-type-chinese-in-latex
\usepackage{CJKutf8}
\AtBeginDvi{\input{zhwinfonts}}
%
\newcommand{\REN}{\begin{CJK*}{UTF8}{gbsn}人\end{CJK*}}
\newcommand{\ren}[1]{{\color{#1}\REN}}



%%%%%%%%%%%%%%%%%%%%%%%%%%%%%%%%%%%%%%%%%%%%%%%%%%%%%%%%%%%%%%%%%%%%%%%%%%%%%%%%
\begin{document}
%%%%%%%%%%%%%%%%%%%%%%%%%%%%%%%%%%%%%%%%%%%%%%%%%%%%%%%%%%%%%%%%%%%%%%%%%%%%%%%%
%%%%%%%%%%%%%%%%%%%%%%%%%%%%%%%%%%%%%%%%%%%%%%%%%%%%%%%%%%%%%%%%%%%%%%%%%%%%%%%%

%%%%%%%%%%%%%%%%%%%%%%%%%%%%%%%%%%%%%%%%%%%%%%%%%%%%%%%%%%%%%%%%%%%%%%%%%%%%%%%%
\section{Intro}
%%%%%%%%%%%%%%%%%%%%%%%%%%%%%%%%%%%%%%%%%%%%%%%%%%%%%%%%%%%%%%%%%%%%%%%%%%%%%%%%



\begin{frame}[plain,t]
	\begin{center}
		%\small
		%
		Clustering in single-cell analyses
		%
		\\[1\baselineskip]
		\small
		RA
% 		\\[1\baselineskip]
% 		\footnotesize
% 		ISS, AS \\ \EMAIL

		\vspace{1cm}

		%

	\end{center}

	\Bottom{
		\scriptsize
		%Support: 
		\hfill
		Nov, 2017
		\\ {\ }
	}
\end{frame}

%%%

\begin{frame}[t]{Single-cell RNA-seq}{Progression of measurement volume}
	
	{\ }

	\centering
	\begin{overpic}[width=\textwidth]{singlecell-scaling}
		\put (0, 0) {\tiny \color{red} img: \cite{SvenssonVentotormoTechmann2017}}
	\end{overpic}
	
	\Bottom{
		\tiny
		\begin{references}
		\item[\cite{SvenssonVentotormoTechmann2017}]
			\bibentry{SvenssonVentotormoTechmann2017}
		\end{references}
	}
\end{frame}

%%%

\begin{frame}[t]{Single-cell RNA-seq}{}

	Common goals \cite{BacherKendziorski2016}
	\begin{itemize}
	\item<1->
		identifying cell sub-populations within a biological condition;
		
		{\only<1>{%
			\begin{itemize}
			\item
				hierarchical clustering, PCA, factor analysis, ZIFA
			\item
				t-SNE
				(clusters from a notion of neighbors)
			\item
				SNN-Cliq
				(rankings of similarities)
			\item
				RaceID
				($k$-means on inter-cell-Pearson's-correlation)
			\item
				SCUBA
				(extract lineage relationships)
			\end{itemize}
		}}%
	\item<2->
		characterizing genes that have differential distributions across conditions;
	\item<3->
		pseudo-temporal ordering and inference;
		
		{\only<3>{%
			\begin{itemize}
			\item
				``Monocle''
				{\tiny
					\bibentry{TrapnellETAL2014}
				}
			\end{itemize}
		}}%
	\end{itemize}
	
	\Bottom{
		\tiny
		\begin{references}
		\item[\cite{BacherKendziorski2016}]
			\bibentry{BacherKendziorski2016}
		\end{references}
	}
\end{frame}

%%%

\def\TITLE{``Wanderlust'' for development trajectory detection \cite{BendallETAL2014}}
\def\REF{
	\tiny
	\begin{references}
	\item[\cite{BendallETAL2014}]
		\bibentry{BendallETAL2014}
	\end{references}
}
\begin{frame}[t]{\TITLE}{}
	Single-cell mass cytometry data, such as
	\begin{itemize}
	\item
		phenotypic proteins,
	\item
		transcription factors,
	\item
		regulatory enzymes,
	\item
		cell-state indicators,
	\item
		activation of regulatory signaling molecules.
	\end{itemize}
	%
	Data
	gathered at once or over time
	---
	only need
	representatives of different developmental stages.
	%
	\underline{No temporal component}.
	
	{\ }
	
	Goal: A typical trajectory
	of cell development
	in the ``phenotypic space''.
	
	\Bottom\REF
\end{frame}
%
%
\begin{frame}[t]{\TITLE}{}
	Input:
	\begin{itemize}
	\item
		cell profiles $X = \{ x_1, \ldots, x_n \} \subset \IR^D$,
	\item
		one {\bf young} cell highlighted as ``initiator cell'',
	\item
		distance function on $X$,
	\item
		ensemble parameters
		\begin{itemize}
		\item
			size of ensemble $n_g$,
		\item
			number of nearest neighbors $k$,
		\item
			subset size $\ell < k$.
		\end{itemize}
	\end{itemize}
	
	Output:
	\begin{itemize}
	\item
		A trajectory ``score'' (measure of maturity) $0 \leq s_i \leq 1$ for each cell $i$.
	\end{itemize}
	
	{\ }
	
	The ``initiator cell'' cell has a score of zero.
	
	\Bottom\REF
\end{frame}
%
%
\begin{frame}[t]{\TITLE}{}
	\adjustbox{valign=c}{%
		\begin{minipage}{0.6\textwidth}
			Global distances are established based on local distances,
			i.e.~from nearest neighbors.
			%
			This is similar to t-SNE.
			
			{\ }
			
			The proposed algorithm includes
			averaging the trajectory score
			over an ensemble 
			of $\ell$-out-of-$k$
			nearest neighbors graphs
			to water down degeneracies.
		\end{minipage}
	}%
	\hfill
	\adjustbox{valign=c}{%
		\begin{minipage}[t]{0.3\textwidth}
			\begin{overpic}[width=\textwidth]{../img/wanderlust-graphxy}
		% 	\begin{overpic}[width=0.3\textwidth,grid,tics=10]{../img/wanderlust-graphxy}
				\put (0, 0) {\tiny img: \cite{BendallETAL2014}}
				\put (58, 90) {\tiny short circuit}
				\put (59, 80) {\tiny graph path}
			\end{overpic}
		\end{minipage}
	}%
	
	\Bottom\REF
\end{frame}
%
%
\begin{frame}[t]{\TITLE}{}
	\adjustbox{valign=t}{%
		\begin{minipage}{0.6\textwidth}
			\begin{overpic}[width=\textwidth]{../img/wanderlust-expression}
		% 	\begin{overpic}[width=0.3\textwidth,grid,tics=10]{../img/wanderlust-graphxy}
				%\put (0, 0) {\tiny img: \cite{BendallETAL2014}}
			\end{overpic}
			
			\tiny
			\centering
			$\rightarrow$ ``wanderlust'' score $\rightarrow$
		\end{minipage}
	}%
	\hfill
	\adjustbox{valign=t}{%
		\begin{minipage}[t]{0.3\textwidth}
			Insights, e.g.:
			
			{\ }
			
			Identify
			new markers 
			of early human B cell precursors.
			
			{\ }
			
			Establish a reference
			\textbf{healthy}
			developmental trajectory
			to
			better understand
			disease.
		\end{minipage}
	}%
	
	\Bottom\REF
\end{frame}

%%%

\def\TITLE{``Wishbone'' for bifurcating development trajectories \cite{SettyETAL2016}}
%
%
\begin{frame}[t]{\TITLE}{}
	Modification of ``wanderlust''
	for trajectories
	with one branching point.
	
	{\ }
	
	\adjustbox{valign=t}{%
		\begin{minipage}{0.4\textwidth}
			{\only<1-3>{%
				\begin{overpic}[width=\textwidth]{../img/wishbone-graphsjt}
				\end{overpic}
				
				\tiny
				All images: \cite{SettyETAL2016}
			}}%
			{\only<4>{%
				\centering
				\begin{overpic}[width=0.7\textwidth]{../img/wishbone-secondev}
				\end{overpic}
				
				The components of the 2nd EV of $Q$ indicate the branch.
			}}%
		\end{minipage}
	}%
	\hfill
	\adjustbox{valign=t}{%
		\begin{minipage}[t]{0.45\textwidth}
			{\only<1>{%
				Preprocessing: projection
				onto the first few \textbf{diffusion maps}
				(coordinates 2--4 constructed from EVs
				of a Markov matrix/process on the original data \cite{CoifmanLafon2006}).
				
			}}%
			{\only<2>{%
				Encode in the matrix $Q_{j t}$
				the excess distance of going
				
				\begin{quote}
					$s \to j \to t$
				\end{quote}
				
				rather than directly
				
				\begin{quote}
					$s \to t$.
				\end{quote}
				
				{\ }
				
				The excess distance is large
				
				{\ } \hfill $\Leftrightarrow$ \hfill {\ }
				
				$j$ and $t$
				are on different branches.
			}}%
			{\only<3-4>{%
				\begin{overpic}[width=0.9\textwidth]{../img/wishbone-distancematrix}
				\end{overpic}
				
				{\ }\vspace{-4\baselineskip}
			}}%
		\end{minipage}
	}%
	
	\Bottom
	{
		\tiny
		\begin{references}
		{\only<1>{
			\item[\cite{CoifmanLafon2006}]
				\bibentry{CoifmanLafon2006}
				\ 
		}}%%
		\item[\cite{SettyETAL2016}]
			\bibentry{SettyETAL2016}
		\end{references}
	}
\end{frame}

%%


\def\TITLE{``SCUBA'' \cite{MarcoETAL2014}}
\def\REF{
	\tiny
	\begin{references}
	\item[\cite{MarcoETAL2014}]
		\bibentry{MarcoETAL2014}
	\end{references}
}
\begin{frame}[t]{\TITLE}{{S}ingle-cell {c}lustering {u}sing {b}ifurcation {a}nalysis}
	
	{\ }
	
	``for extracting lineage relationships 
	from single-cell gene expression data
	and modeling [..] cell differentiation.''
	
	
	\Bottom\REF
\end{frame}

%%%%%%%%%%%%%%%%%%%%%%%%%%%%%%%%%%%%%%%%%%%%%%%%%%%%%%%%%%%%%%%%%%%%%%%%%%%%%%%%%
%\section{Extra}
%%%%%%%%%%%%%%%%%%%%%%%%%%%%%%%%%%%%%%%%%%%%%%%%%%%%%%%%%%%%%%%%%%%%%%%%%%%%%%%%%
%
%
\newcounter{finalframe}
\setcounter{finalframe}{\value{framenumber}}
% Backup frames follow
%
%
%\begin{frame}
%	Appendix
%\end{frame}
%
%%
%
%\begin{frame}
%	%
%\end{frame}
%
%
% FINAL SLIDE
\setbeamercolor{background canvas}{bg=black}
\begin{frame}[plain,b]
	\hfill
	\tiny
	\color{gray}
	this slide is intentionally left blank
\end{frame}
\setbeamercolor{background canvas}{bg=white}


%%%%%%%%%%%%%%%%%%%%%%%%%%%%%%%%%%%%%%%%%%%%%%%%%%%%%%%%%%%%%%%%%%%%%%%%%%%%%%%%%
%\section{Bibliography}
%%%%%%%%%%%%%%%%%%%%%%%%%%%%%%%%%%%%%%%%%%%%%%%%%%%%%%%%%%%%%%%%%%%%%%%%%%%%%%%%%

\scriptsize
\bibliography{../../../r/refs}


%%%%%%%%%%%%%%%%%%%%%%%%%%%%%%%%%%%%%%%%%%%%%%%%%%%%%%%%%%%%%%%%%%%%%%%%%%%%%%%%
\setcounter{framenumber}{\value{finalframe}}
\end{document}
%%%%%%%%%%%%%%%%%%%%%%%%%%%%%%%%%%%%%%%%%%%%%%%%%%%%%%%%%%%%%%%%%%%%%%%%%%%%%%%%
%%%%%%%%%%%%%%%%%%%%%%%%%%%%%%%%%%%%%%%%%%%%%%%%%%%%%%%%%%%%%%%%%%%%%%%%%%%%%%%%

